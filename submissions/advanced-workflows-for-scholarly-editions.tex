\documentclass[11pt,a4paper]{article}
\usepackage{lac2014}
\sloppy
\newenvironment{contentsmall}{\small}

\usepackage{fontspec}
\usepackage{polyglossia}
\usepackage{microtype}

\title{Version Controla and Collaborative Workflows for Scholarly Editions}

%see lac2012.sty for how to format multiple authors!
\author
{Urs LISKA \and Janek WARCHOŁ
\\ openLilyLib.org / lilypondblog.org
\\ info@openlilylib.org
}



\begin{document}
\maketitle


\begin{abstract}
\begin{contentsmall}
Plain text based tools and version control offer unique and fundamental advantages over
graphical approaches when preparing scholarly editions.
But it is difficult to convince musicologists, engravers and publishers getting their hands
dirty with source code.

This paper presents our experiences with two complex edition projects and outlines
how we are striving towards an ideal working environment by extending our favorite tools LilyPond
and LaTeX with specific solutions.

\end{contentsmall}
\end{abstract}

\keywords{
\begin{contentsmall}
Musical engraving, scholarly editions, version control, work-flows.
\end{contentsmall}
}

\section{Introduction}

\section{Fundamentals}
% The theoretical foundation.
% As short as possible.
% Link to the essay on Scores of Beauty

\subsection{Plain Text}

\subsection{Version Control}

\section{Collaboration}
% Hands-on experiences from the Fried songs edition

\subsection{Interaction Between Editor and Engraver}
% It is actually a hot topic in scholarly discussion
% to what extent the editor should also be an engraver.
% Scholarly editions usually let the editor only provide
% a model, while an engraver from the publishe prepares the
% score. By contrast commercial publishers usually require
% the editor to prepare a near-printable score document.

\subsection{Music entry}

\subsection{Proof-reading}

\subsection{Beautification}

\subsection{Maintainability After Publication}


\section{A New Project: Crowd Editing}

\subsection{Interaction and Peer Review}

\subsection{Using Programming to Simplify the Editors' Life}
% Mabe this subsection is too OT or project specific.
% This should be the first to be dropped if it's too long.


\section{Extending the Tools With Musicological Perspective}
% I'm not sure if this section will remain in the paper.
% It's slightly OT and might eat up too much space,
% OTOH I wouldn't want to miss it completely.
% Probably there should be a few notes here and there.
% Or maybe a short 'prospects' subsection in 'Conclusions'

\subsection{Graphical Editing for Frescobaldi}

\subsection{\texttt{\textbackslash annotate}}

\subsection{Embracing Music Encoding}


\section{Conclusions}

Concluding text.

\section{Acknowledgements}

Our thanks go to \ldots .

\end{document}
