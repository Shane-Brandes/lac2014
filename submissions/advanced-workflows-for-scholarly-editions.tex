\documentclass[11pt,a4paper]{article}
\usepackage{lac2014}
\sloppy
\newenvironment{contentsmall}{\small}

\usepackage{fontspec}
\usepackage{polyglossia}
\usepackage{microtype}

\title{Advanced Workflows for Scholarly Editions Based on Plain Text}

%see lac2012.sty for how to format multiple authors!
\author
{Urs LISKA \and Janek WARCHOŁ
\\ openLilyLib.org / lilypondblog.org
\\ info@openlilylib.org
}



\begin{document}
\maketitle


\begin{abstract}
\begin{contentsmall}
Plain text based tools and version control offer unique and fundamental advantages over
graphical approaches when preparing scholarly editions.
But it is difficult to convince musicologists, engravers and publishers getting their hands
dirty with source code.

This paper presents our experiences with two complex edition projects and outlines
how we are creating an ideal working environment by extending our favorite tools LilyPond
and LaTeX with specific solutions.

\end{contentsmall}
\end{abstract}

\keywords{
\begin{contentsmall}
Musical engraving, scholarly editions, version control, work-flows.
\end{contentsmall}
}

\section{Introduction}

This is a model document. Don't use fonts smaller than this
one (Times 11), and don't forget to leave it in A4 (21 x 29.7~cm).

\section{Section}

Text\footnote{Text of note.}, note at end of page.


\subsection{Subsection}


 
\subsubsection{Subsubsection}

Text of the subsubsection.
Text of the subsubsection.
Text of the subsubsection.
Text of the subsubsection.
Text of the subsubsection.
Text of the subsubsection.
Text of the subsubsection.
Text of the subsubsection (see Table~\ref{table1}).

Text of the subsubsection.
Text of the subsubsection.
Text of the subsubsection.
Text of the subsubsection.
Text of the subsubsection.
Text of the subsubsection.


\begin{table}[h]
 \begin{center}
\begin{tabular}{|l|l|}

      \hline
      Software & Features\\
      \hline\hline
      AA & Harddisk-Recording\\
      BB & MIDI Sequencing\\
      CC & Score Notation\\
      \hline

\end{tabular}
\caption{Example}\label{table1}
 \end{center}
\end{table}


Text of the subsubsection.
Text of the subsubsection.
Text of the subsubsection.
Text of the subsubsection.


\section{Section}

Text. Text. Text. Text. Text.
Text. Text. Text. Text. Text.

Text. Text. Text. Text. Text.
Text. Text. Text. Text. Text.
Text. Text. Text. Text. Text.
Text. Text. Text. Text. Text.
Text. Text. More text. Text. Text.

\section{Section}

Text. Text. Text. Text. Text.
Text. Text. Text. Text. Text.
Text. Text. Text. Text. Text.
Text. Text. Text. Text. Text.
Text. Text. More text. Text. Text.
Text. Text. Text. Text. Text.

Text. Text. Text. Text. Text.
Text. Text. Text. Text. Text.

\section{Conclusions}

Concluding text.

\section{Acknowledgements}

Our thanks go to \ldots .

\end{document}
