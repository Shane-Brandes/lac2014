\documentclass[11pt,a4paper]{article}
\usepackage{lac2014}

% Style the document.
% For any further use of the document (e.g. in the context of proceedings)
% this \usepackage can simply be removed.
% Packages that may be necessary to display the _content_ of this paper
% are called directly from within this file.
\usepackage{ulCFPstyles}

% More packages needed for compiling this document
\usepackage{hyperref}

\sloppy
\newenvironment{contentsmall}{\small}

\usepackage{fontspec}
\usepackage{polyglossia}
\usepackage{microtype}

\usepackage{xcolor}

\title{A New Concept of Recording Classical Music}

%see lac2012.sty for how to format multiple authors!
\author
{Urs LISKA
\\ mail@ursliska.de
}



\begin{document}
\maketitle


\begin{abstract}
\begin{contentsmall}
This paper reconsiders the “Arranger Window” concept of \textsc{daw} programs from the perspective of recording classical music.
I will present a new concept of recording classical music with the main concern being the organization of the recorded material and the preparation of an editing plan.
For this I will outline some ideas of a possible approach to designing a software application.

Specific stress is put on plain text storage and the use of version control for project management, offering perspectives for collaborative workflows.

\end{contentsmall}
\end{abstract}

\keywords{
\begin{contentsmall}
Music production, classical music, workflows.
\end{contentsmall}
}

\section{Introduction}

As a technically inclined musician I have always been interested in recording processes, from pubertal experiments on a four-track tape deck until today where music production on the computer is ubiquitous.
But in recent years I had the opportunity (as a pianist) to produce a large-scale series of professional recordings, resulting in eight CDs, recorded in more than 30 studio days over several years.
The recording producers entrusted me with the complete recordings along with their annotated scores so I could judge the first edits and very concisely express wishes and suggestions for improvements. 
For some of the recordings I actually did the first edits myself.
This way I had around 70 hours of recorded material to digest, which gave me more than enough reason to reflect on the workflows imposed by the usual \textsc{daw} tools.

This experience convinced me that the metaphor of a multi-track magnetic tape recorder isn't really appropriate for recording classical music.
In this paper I'm going to outline a new approach to that task which is based on the initial question: “What if the takes knew about their content instead only about their length?”

I'm going to make suggestions for a possible \textsc{gui} application, although I currently don't (or rather can't) have any plans to implement them.
However, discussing them publicly may generate interesting insights and maybe motivate people to seriously think about making use of them.
And given enough interest from people with the necessary complementary skills I might even be talked into starting a new project.

\section{Two Recording Paradigms}

Of course there are more shades to the matter  than simply speaking of “pop” and “classical” music.
And of course there is a variety of approaches how to produce recordings that don't match the black \& white labeling suggested by such a separation.
So please bear with me if I'm occasionally over-generalizing things.
But I think there basically are two main paradigms of recording music, and for simplicity's sake I will call them the \emph{pop} and the \emph{classical} approach in this paper.

\subsection{Recording Pop Music and the Multi-Track Tape Metaphor}

The \emph{pop} approach relies on the metaphor of a multi-track magnetic tape, represented by an \emph{Arranger Window} in virtually all \textsc{daw} programs I know of.
The tape reel is represented as a preexistent \emph{timeline}, and each new “take” is recorded to its final position on it, adding to anything else that has been already recorded.
That we \emph{can} move takes around or copy, multiply and edit them is an additional feature that doesn't make our approach obsolete.
Different layers (voices/instruments) of the arrangement can be recorded sequentially as “overdubs” on multiple “tracks” representing those on the vintage tape.

\subsection{Recording Classical Music}
 
The \emph{classical} approach rather imitates a stereo magnetic tape treated with scissors.
The complete arrangement is recorded to one track and there are no overdubs.
Instead the editor selects parts from the takes and glues them together, creating a seemingly continuous recording.

While this process results in a series of segments on a horizontal timeline too, there are significant differences to the previously described set-up:
The timeline is completely flexible and doesn't have a direct relation to the musical content.
And there isn't the vertical stack of multiple tracks, \emph{conceptually} there is only one.

Obviously it's perfectly possible to produce \emph{classical} recordings with tools adhering to the \emph{pop} paradigm, but it's far from natural, and there could be fundamental improvements on the way we conceive this specific task. 

\section{A New Approach to Managing Recorded Material}

In the process of preparing an editing plan for a recording of classical music the task of a recording producer boils down to keeping track of recorded takes and choosing the best take for each section of the composition.
If the recording session has been complicated and yielded a large number of takes this house-keeping can be a tedious and error-prone process.
Usually we make do with the more or less sketchy pencil annotations in the score which list starts and stops of takes, probably some notes about good or bad sections/points and maybe already some ideas about transitions between takes.
But it's tedious (to say the least) to \emph{reliably} look up usable takes for any given measure from such a score.
In order to ensure nothing has been missed we should rather document the material quite thoroughly.
First we need to list the recorded ranges of all takes (including differing subtakes) in order to know which takes actually cover a given moment.
If we don't want to listen to the same recordings over and over again we also have to document our evaluation of the takes as detailed as possible.
But even this will merely give us pointers to the complete takes to consider while we still have to “use” the waveform display to navigate through them.

So the question is: If we're going to document our material anyway, why don't we simultaneously “teach” the takes to know about themselves?
In the following sections I'll demonstrate an approach to exploiting this question, and I will start rather abstractly, proceeding to concrete implementation ideas later on.

\subsection{Takes Know About Their Content}

The core idea of this paper is making the takes completely aware of themselves.
If the takes “know” about the segments of the composition they contain and the editor's rating they can “answer” the question: “Which options do I have for an inserting edit in measure 17?”
In order to achieve this I suggest a switch of perspective and don't deal with \emph{Regions} as segments of audio files but rather with \emph{Takes} as segments of the composition.

So in a way we require our project to be aware of the composition.
To get a usable application we should however ensure not to require anybody to enter a detailed project structure before starting to work -- this would surely be a show-stopper.
Instead it should be possible to simply start with the number of measures%
\footnote{Or any other unit or numbering scheme, e.\,g.\ rehearsal marks, systems or pages.}
and refine the representation of the musical structure along the way -- while doing the housekeeping we have to do anyway.
The project will then inherently learn about the composition with each added piece of information.

\subsection{Annotating Takes}
Instead of paper or a separate text or spreadsheet document we can simply use the take files themselves to store their documentation in.

The very least a take has to know is the covered range in musical terms, along with a reference to the audio file it is using.
This is all we have to provide to get started with the take, so the new approach won't actually impose any overhead.
But the more information we feed the take -- as the documentation of our musical analysis -- the more we can benefit from the new approach.
I'd estimate this expects 15\,\% overhead in thorough documentation but gives back 150\,\% through increased efficiency.

A take contains information about the musical structure of its content, the relation of this content to the audio file and finally our rating of the recording.
Information on the musical structure is stored and synchronized globally, so if we enter any information (such as e.\,g. a change in time signature, a rehearsal mark etc.) in a take it will be automatically propagated to the project and consequently to any other takes covering that section of the composition.
Initially positions in the audio files will be estimated by interpolating linearly, but each new marker set in any take file will extend the coverage and improve future estimates.
This information will be used to locate playing positions in the takes, and in the end we will be able to see and listen to all alternative takes for any given musical moment, without having to look up more or less reliable annotations in the score.

\subsection{Storage and Version Control}

I suggest storing the information about takes in individual text files, one file for each take.
This is in line with the concept of the takes being self-sufficient, and in particular it is practical for using version control to manage projects.
An application implementing my concept should transparently use plain text files and version control as its storage mechanism.

I see several advantages in using Git in the storage concept:
\begin{itemize}
\item Undo/redo cannot only be \emph{unlimited} but also completely \emph{selective}.
\item There is an exhaustive \emph{project history} readily available to be presented in any desired form.
Commit messages are also a natural place to (optionally) store meta information.
\item \emph{Branching} provides a way to always work in the context of an implicit session.
There is no need for “autosave”, because all changes are stored immediately to disk, so a program crash doesn't lose \emph{any} information.
Instead of “saving” at the end of the session the current branch is merged into master.
\item I'm sure one can make good use of branching to design workflows with “named sessions” (e.\,g.\ for trying out some hacks or to switch working context temporarily).
\item And of course this widely opens the door for \emph{collaborative workflows} where for example the musicians can evaluate takes while the editor is preparing his editing plan.
\end{itemize}

However, this has to be absolutely transparent.
The ordinary user should not be required to learn \emph{any} Git syntax because such a requirement would be a massive obstacle getting acceptance for the new concept.

\medskip
Instead of inventing a new plain text file format I suggest using LilyPond%
\footnote{\url{http://www.lilypond.org}}
files as storage format.
We are talking about describing and annotating musical structures, and LilyPond offers a concise input language for just this, which is also very suitable for version control.
It is highly customizable through it's built-in Scheme extension, so we can for example accomodate the annotation contexts, and we could even use LilyPond itself to “engrave” an editing plan as a score.

\section{Components of an Application}

By now it seems possible to think about implementing a \textsc{gui} application to annotate takes.
But in my eyes this would only make sense if it's possible to create a professional tool for professional users, anything else seems like a waste of time.
Before I finish this paper with some thoughts how such an application look like in general and where it could fit in the “workbench” of a professional recording producer I'm going to describe the core components that such an application would have to have.

\subsection{Annotating a Single Take}
The core element of our application is an editor for individual takes.
This is what we'll spend most of our time with, particularly during the initial evaluation of the recorded material but also later at any time.
Therefore this shouldn't be a dialog but a dockable panel which is permanently visible and simply gets updated when a new take is selected.
It contains a zoomable graphical representation of the audio in the take, presumably as a waveform or volume graph display, as well as some elements like marker indicators and editors.
Playback controls should cover the usual range of features like moving forward and backward, listening to ranges, start from marker positions etc.

This representation of a take is covered by a “timeline” representing the musical structure, that is it doesn't (primarily) displays a time code but rather barlines rests and time signature, similar to a percussion staff.
Rehearsal marks, tempo indicators and more might easily be added to this.
Initially this musical structure consists of evenly spaced measures with a default time signature, but during work the editor can add relevant information such as time signature changes.
And he can link the musical structure to positions in the audio files with an interface similar to setting markers.
Through this the program also gets an inherent understanding of tempo changes allowing it to interpolate positions for playback.
The timeline should visually distinguish between positions \emph{set explicitely} and \emph{interpolated}.

(There is one little detail I'd like to point out here: By trimming a take to the \emph{usable} range one can effectively kick the useless material out of sight.
No need anymore to re-listen to false starts or page-turning interruptions etc.)

Information edited in an individual take will be propagated to the whole project and may be averaged with information from other takes.
This will deepen the project's knowledge about the musical structure and gradually improve the accuracy of position interpolations.
This means that if you start to edit a take covering music which was already annotated in another take the program already knows about changes in time signatures and tempo.

But of course the most relevant information added to takes is our evaluation.
It should be possible to rate any range or spot numerically, with some special values available form marking something as “unusable”, “usable as base material”, “suggested” or ”to be used” and “editing plan”.
Apart from numerical or tabular display the waveform representation could visualize this through differently colored background or a similar way.
Adding ratings should be accessible through right-clicking on the waveform or through keyboard shortcuts while playing back.
Apart from the ratings one can add arbitrary verbal comments to the take as a whole, specific spots or regions or explicit ratings.

It should be possible (but I'm not sure yet if it's worth the effort) to compile a score fragment with LilyPond, representing the musical structure and displaying the rating and comments in some way.

\subsection{Project Overview}
The “Project View” of this initial application should be very much list-like.
Graphical solutions integrating into a “timeline”-like representation of the composition and much more are considerations for the last section of this paper.
So we start with something like a traditional “Region List”, but with the benefit of having superior information available for sorting, highlighting or filtering items, namely the information on the musical content and our ratings.

While in traditional \textsc{daw}s takes are initially sorted by number, that is by their recording order -- which usually is only vaguely related to the musical structure of the recorded piece.
By default our application should sort takes by their starting point, but we could also sort by overall rating, length of usable ranges or whatever we consider useful.
Filtering can be applied so we can hide takes based on their content or their rating.
For example we could -- while editing a single take -- see a list of all alternative takes containing the spot the playback cursor currently is, reversly sorted by rating.
I won't list more ideas here (because that's something for the actual application design), but I'm sure you get the idea how this approach can dramatically increase productivity for the main task a recording producer has for editing classical music: preparing the editing plan.

\subsection{Editing Plan}
Conceptually an editing plan is a list of pairs with musical sections and references to takes.
Traditionally takes are only referenced as numbers because that's all we have to refer to.
But in our application we have significantly more: links from musical moments to positions in the audio files the takes are based on.
The precision of these links depends on the accuracy of our annotations, but generally one can expect that particularly the points actually interesting for transition between takes have been thoroughly examined.
Suggesting to add exact links to these points surely isn't much overhead, particularly given the advantage you gain by it.

With the knowledge our application has of the material we can easily display a list of takes that have ratings of at least “suggested”.
But even more interesting is the reversal: a list of musical ranges that \emph{don't} have takes with sufficient rating.
With these options we always have a clear picture of the progress of our editing plan, we see which parts of the composition still have to be resolved, and for the other parts we can see which alternative takes we may choose from.

In addition to a tabular representation this can be presented in a \textsc{ui} element quite similar to an arranger window.
The horizontal axis represents the timeline but is of course primarily structured musically.
The topmost “track” is reserved for the final editing plan, then there are tracks (or groups of tracks) for “to be used” or “suggested” takes, followed by a last area for all other takes.
Takes are presented as blocks (by default without waveform display but colored according to the rating) placed at their position in the piece.
This way one has a complete overview of the available recorded material, structured by relevance and quality through the assignment to tracks and by the color-coding.

When we're finished with our preparation -- that is the “editing plan” track is seamlessly filled with take fragments -- the editing plan can be exported: As a list with transition points and references to takes and positions in them, and possibly as a score (compiled by LilyPond) displaying the same information.
Additionally it would be nice to be able to export to \textsc{daw} programs' project files so one could for example open the result in Ardour and have the relevant takes already in the arranger window, ready to be finally edited.

\subsubsection{Transition Editor}
There is one more topic to consider, which would be more suitable for the final section of this paper but which is necessary to make a usable initial application: a \emph{Transition Editor}.

One crucial point when preparing an editing plan is not only to select usable takes for a given musical moment or range but also check if any considered transition between takes is actually possible.
Takes have to be compatible with respect to tempo, dynamics, internal balance and “atmosphere” (reverberation trails are particularly tricky) -- and this can only be tested by actually trying a transition.
Without such a feature even an initial application wouldn't be really usable.

Therefore I suggest to implement a rather simplistic Transition Editor which allows to position two takes relatively and offer simple options to edit the cross-fade.
From my experience this will be sufficient in most cases to judge if a given transition can be done later with a professional tool -- please refer to the last section of this paper for my thoughts on the future of this functionality.

By moving this to a dedicated tool we can avoid usability issues one usually has when working in an arranger window, particularly we can benefit from the fact that we're not bound by keeping in sync with a timeline.
So we don't have to scroll and zoom around to find and pull interesting takes to our current timeline position.
Instead (as we know about the links between musical structure and playback positions in the audio files) we can simply \emph{select} a desired take, which is then immediately present pre-cut at about the right point.
Imagine a dialog where you have a musical moment (e.\,g. “M.\,128, 3rd beat”) as the central reference point.
From a dropdown list you can then choose (for the “left” or “right” part of the transition) from all applicable takes, sorted by rating.
All this is self-contained and completely independent from the notion of a timeline.
The result of such an edit will simply be a usable transition which can be used in the editing plan (or put on a stash of usable transitions for later consideration).
If we have edited all transitions that are necessary to join the take fragments in the editing plan everything is there to render a preliminary first edit of the recording -- which will in most cases already be usable as a nice reference.
How this is actually managed and made accessible for a user interface will be a matter of concrete application design.

\section{The Bigger Picture}
While an application as outlined in the previous section could already be tremenduously efficient for managing classical recordings it is still far from being able to replace dedicated \textsc{daw}s like Ardour, ProTools or Pyramix from the perspective of a professional recording producer.
But anything less than aiming at a professional tool for professional users doesn't seem to make sense to me.
Therefore it will be necessary to carefully consider how to achieve professional quality.

There are new thoughts in my approach and surely some unique structural and interface concepts, but of course many of the problems at hand have already been solved perfectly.
For example I don't see a point in reinventing a versatile waveform display.
And there has of course been so much research in Digital Signal Processing that it seems stupid trying to do anything in this area better than existing solutions, without any prior experience.
So the question is if and how it is possible to reuse existing solutions, to integrate with existing applications or at least provide tools and workflows for a seamless cooperation with existing applications.

Probably the nicest solution would be to actually create an application that contains everything one needs.
This would make it possible to benefit from the new interface approach and complete a production within one environment.
Take the Transition Editor as an example: If it would be possible to tweak this to professional level it would provide a unparalleled tool to edit classical recordings.
But in order to achieve that it will be indispensable to reuse existing technology, either from applications like Ardour or through libraries and toolkits.

If one decides that this isn't the way to go one would have to see if it's possible to integrate the concepts into existing applications, for example by offering an alternative to the arranger window and providing the necessary additional tools.
This way one could simply \emph{use} existing tools like waveform display, audio core, editors etc. -- and maybe at the same time offer the new tools to be used in other contexts in the existing application.
But it may well be that such an approach would even be more complex than inventing a complete application.

A last approach -- and maybe not the worst -- would be to have the new application as a standalone program but offer tools to integrate workflows with other programs.
For example my application would work on existing audio files which are located anywhere and “imported” through a simple symlink.
If that symlink pointed to the audio folder of another application one could simply do the recording in the other application and the newly recorded takes are automatically added to the project.
The editing plan could then be prepared in the new application and the final editing be done in the other \textsc{daw}.
This approach has the main advantage that the recording producers can use whatever tools they are used to (or have available) -- provided our new application is able to run on the same OS or can use a shared storage device it can link to.

\subsection{Development Platform}
Of course this project can be realized in nearly any conceivable way, but if I were to start it I would start with a PyQt application.
Having some experience contributing to the Frescobaldi%
\footnote{\url{http://www.frescobaldi.org}}
LilyPond editor I think this is a very suitable platform for this kind of project -- and particularly the one I'm familiar with.

The \textsc{ui} elements needed for this application could completely be reused in a greater context of a \textsc{daw}-like program, therefore I would try to design them in a way that they can simply be \emph{used} as widgets in alater program too.

\subsection{Final Considerations}
There are a few topics that have to be considered

Other processes involved but not my main interest:

\begin{itemize}
\item Recording music
\item Mixing music
\item Actual editing (?)
\item Exporting (e.g. to CD)\\
	The least would be to export an Ardour project file
\item Git workflows, particularly collaborative ones.
\end{itemize}

\section{Conclusions}

State only here that the Arranger Window is completely inappropriate?

\begin{quote}


In a classical recording we only need one track at a time.
Actually we are recording on multiple tracks (each sound source has its own track) -- but only as a means of separating sound sources up to the final mixdown, in general we don't have any interest in acccessing and displaying them separately during the editing process.

Usually the editor uses additional tracks in the Arranger Window as a kind of dashboard to juggle with alternative takes until he's decided which ones to use for the final edit.
But while this is \emph{possible} it is quite \emph{counterproductive} for the major tasks when preparing a classical recording.
\end{quote}

\section{Acknowledgements}

Our thanks go to \ldots .

\end{document}
