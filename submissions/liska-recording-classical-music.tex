\documentclass[11pt,a4paper]{article}
\usepackage{lac2014}

% Style the document.
% For any further use of the document (e.g. in the context of proceedings)
% this \usepackage can simply be removed.
% Packages that may be necessary to display the _content_ of this paper
% are called directly from within this file.
\usepackage{ulCFPstyles}

% More packages needed for compiling this document
\usepackage{mdwlist}
\usepackage{hyperref}

\sloppy
\newenvironment{contentsmall}{\small}

%\usepackage{fontspec}
%\usepackage{polyglossia}
%\usepackage{microtype}

\usepackage{xcolor}

\newcommand*{\term}[1]{\textcolor{blue}{\emph{#1}}}
\title{A New Concept of Recording Classical Music}

%see lac2012.sty for how to format multiple authors!
\author
{Urs LISKA
\\ ul@openlilylib.org
}



\begin{document}
\maketitle


\begin{abstract}
\begin{contentsmall}
A reconsideration of the \term{arranger window} concept of \textsc{daw} programs
from the perspective of recording classical music.
A new concept of recording classical music with the main concern
being the organization of the recorded material and the preparation of an
editing plan is explored and some possible approaches towards designing a
software application are outlined.

Specific stress is put on plain text storage and the use of version control for
project management, offering perspectives for collaborative workflows.

\end{contentsmall}
\end{abstract}

\keywords{
\begin{contentsmall}
Music production, recording, classical music, workflows.
\end{contentsmall}
}

\section{Introduction}

As a technically inclined musician I have always been interested in recording
processes, from adolescent experiments on a four-track tape deck until today, where
music production on the computer is ubiquitous.
But in recent years, I had the opportunity and reason to reflect on the workflows
imposed by the usual \textsc{daw} tools in a more substantial way.
As a pianist I have produced a considerable number of professional recordings,
totaling eight CDs, recorded in more than 30 studio days over several years.
The recording producers entrusted me with the complete recordings along with
their annotated scores so I could judge the first edits and express my wishes
and suggestions for improvements very concisely. 
For some of the recordings I actually did the first edits myself with
\emph{Ardour}.
With that experience I had around 70 hours of recorded material to sort, evaluate and
process.
This experience convinced me that the fundamental concepts of today's
\textsc{daw} programs are really designed for producing \emph{pop} music.

Therefore a new approach to the task of managing
recorded material from the perspective of recording \emph{classical} music was developed and offered here.
The central idea is to shift focus from a fixed time-line in an arranger
window to the concept of self-contained \term{takes} that are aware of their
content instead of only their length.
I'll outline ideas how these new ideas could be reflected in a \textsc{gui}
application, although I currently don't have (or rather can't afford) any
implementation plans.
After such a new method is properly developed one could then consider to which extent such an
application could OR should be aimed at being a \emph{replacement} for existing
tools or how it could be \emph{integrated} in an efficient toolchain.

By sharing these ideas publicly interesting insights and
feedback may be generated, and maybe this discussion can motivate people to seriously think about
making use of them.
And given enough interest from people with the necessary complementary skills a new project might be worthwhile to start in order to implement such a editing technique.

\section{Two Recording Paradigms}

Of course there are more shades to the matter  than simply speaking of “pop” and
“classical” music.
And of course there is a variety of approaches how to produce recordings that
don't match the black \& white labeling suggested by such a separation.
So please bear with me if I'm occasionally over-generalizing things.
But I think there basically are two main paradigms of recording music, and for
simplicity's sake I will call them the \term{pop} and the \term{classical}
approach in this paper.

\subsection{Recording Pop Music and the Multi-Track Tape Metaphor}

The \emph{pop} approach is based on the metaphor of a multi-track magnetic tape,
reflected by an \term{arranger window} in virtually all \textsc{daw} programs I
know of.
The tape reel is mapped to a \term{timeline}, and each new \term{take} is recorded to
an absolute position on it, adding to or replacing anything that has already
been recorded.
Different layers (voices or instruments) of the arrangement can be recorded
sequentially as \term{overdubs} on multiple \term{tracks} representing those on the
vintage tape.
In the \textsc{daw} implementation of this concept we can move takes around or copy,
multiply and edit them, but everything still takes place in the context of the
absolute timeline.

\subsection{Recording Classical Music}
 
By contrast the \emph{classical} approach imitates a magnetic tape
treated with scissors and adhesive tape.
The complete arrangement is played simultaneously and recorded to one track,
without overdubs.
Instead of choosing from the available (or re-recording) \emph{parallel} takes
the editor cuts out sections from the recorded takes and glues them together
\emph{consecutively}, creating a seemingly continuous recording.
In analog days such a cutout of a magnetic tape was always treated as a single entity,
regardless of how many tracks were recorded through discrete recording heads.

While this process results in a series of segments on a horizontal timeline too,
there are significant differences from the previously described set-up.
In particular the absolute position of any given take isn't defined until the
very end of the editing process when it is finally selected and glued to the
preceding segment.
The reference point for any given snippet is \emph{not} relative to the project
but to the preceding snippet.

Translating this concept to the digital domain by using the \term{arranger window}
metaphor introduces fundamental issues which current \textsc{daw}s struggle to find
viable workarounds.
The discrepancy becomes apparent when a take selection has to be replaced or changed
so its length is modified: All regions that follow down the timeline have
to accomodate that change, scrupulously taking care not to break any existing work.
Of course today's tools offer assistance with these problems, but the fundamental
issue remains.

Another difference is that \emph{conceptually} we don't need a vertical stack of
tracks but only one%
\footnote{Of course the source material will be recorded to discrete tracks on
disk, but from a user's perspective it's only one track in most situations.}.
The presence of multiple tracks that \textsc{daw}s offer is actually a waste of
screen estate for the recording producer working on classical music.
Usually she \emph{does} use these tracks as a kind of dashboard in which to
experiment with alternative takes for a given moment%
\footnote{Many \textsc{daw}s offer the option to consecutively record multiple
takes to parallel tracks. But this doesn't avoid the problem that these takes
are still absolutely positioned on the timeline.}, but it just isn't a really
natural approach to the task.

Of course reality proves that it \emph{is} perfectly possible to produce classical
recordings with tools adhering to the pop paradigm. Current tools are very
smart in dealing with the conceptual flaws.
But it is possible to design better workflows by developing an application that
takes the \emph{natural} perspective right from the start.
Treating takes as entities and referencing takes to each other instead of to a
timeline are the foundations to which a new approach to managing the recorded
material will add a convenient interface.


\section{A New Approach to Managing Recorded Material}

While preparing an editing plan the task of a recording producer boils down to
keeping track of recorded takes and choosing the best take for each section of
the composition.
If the recording session has been complicated, this housekeeping can be a
tedious and error-prone process.
Usually the recording session yields an annotated score with entries for take
starts and usually (but not equally reliably) also where takes end.
Additionally the recording producer will make notes about good or bad takes,
problematic or particularly good spots in individual takes, and sometimes also
suggestions about concrete transitions between takes.

If the recording session has been easy it can be a straightforward process to
create an editing plan from that.
But if it left the producer with a three-figure number of takes it's very
tedious (to say the least) to \emph{reliably} look up usable takes for any given
measure from such a score.
First of all she has to determine \emph{all} takes covering the range in
question, which can be cumbersome because she has to ensure not to miss anything
which started earlier.
Then it's crucial to have the takes evaluated and for this evaluation to be
documented properly -- in order to avoid having to listen to the takes over and
over again.
But even this will merely give pointers to the takes as a whole -- while
navigating to them to find the actual position of the given musical moment is
still up to the user.

So the fundamental idea of my proposed new concept is:
If we're going to document our material anyway, why don't we simultaneously
“teach” the takes about themselves?
In the following subsections I'll elaborate on this conceptual layer, leaving out
ideas for concrete implementations for a later section.

\subsection{Takes Know About Their Content}

The core idea of this paper's approach to recording classical music is making
the takes completely aware of their musical content and the editor's rating.
In order to achieve this I suggest a switch of perspective and not to deal with
\term{regions} as parts of recorded audio files but rather of \term{takes} which
in turn are conceived as parts of the composition.
For the whole process of analyzing and evaluating the recorded material we don't
need the arranger window with its fixed timeline at all.
By contrast we're only interested in the selection and ordering of takes as well
as in determining usable transitions between them.
Instead of \emph{placing} regions on a timeline we're going to \emph{chain}
takes and define their transitions.
Therefore our focus (and that of a new application) will be on working with
individual takes.

As a consequence we also require our project to be aware of the composition,
which is an additional layer and therefore some overhead the producer has to
maintain.
But I'm sure this investment instantly pays off and isn't a real issue.
To get started with a project the basic requirement is to provide not more than
the total number of measures%
\footnote{Or any other unit or numbering scheme, e.\,g.\ rehearsal marks,
systems or pages.}.
The producer will refine the representation of the musical structure along the
way -- while doing the housekeeping she has to do anyway.

With each added piece of information the project will inherently learn about the
composition.
And with that knowledge it will increasingly be able to assist the producer in
managing the recorded material.

\subsection{Annotating Takes}
Traditionally the producer documents the material in some sort of list, on paper
or in a (text or spreadsheet) document.
Instead of this I suggest to implement dedicated \term{take} objects that
contain all relevant information about a take.

The fundamental data a \term{take} object has to contain is the covered range in
musical terms, along with a reference to the audio file it is contained in.
This is all the producer has to provide for all takes to get started with a
project, so the new approach won't actually impose any significant overhead to
the user.
We can further narrow this down by storing the \emph{usable} recorded range (as
opposed to the total range), along with pointers to the actual positions in the
audio files.
This will effectively hide away any unusable material so we don't have to bother
with it anymore.

While evaluating a take we can add more information on the musical structure to
the take object.
We can add metadata-like time signature changes and particularly markers linking
positions in the music (e.\,g.\ measures) to positions in the audio file.
Each new entry will increase the take's knowledge about the musical structure
and will improve the interpolation of other positions.
This information is synchronized globally, so adding information to a take will
increase the whole project's data coverage.
An application will be increasingly able to locate positions in audio files that
correspond to a given musical moment.
Concretely this will allow the producer to ask for alternative takes, and the
application can present all takes covering the given range and navigate to the
respective positions for immediate playback.

\subsection{Storage and Version Control}

I suggest storing the information about the \term{take} objects in individual
text files, one file for each take.
An application implementing my concept should transparently use plain text files
and version control as its storage mechanism, for which I see several
advantages%
\footnote{What I've written with regard to scholarly editing of scores
(\url{http://lilypondblog.org/2013/07/plain-text-files-in-music/}) applies
equally to the current context.}:
\begin{itemize}
\item Undo/redo cannot only be \emph{unlimited} but also completely
\emph{selective}.
\item There is an exhaustive \emph{project history} readily available to be
presented in any desired form.
Commits are also a natural place to store meta information in their messages.
\item \emph{Branching} provides a way to always work in the context of an
implicit session.
There is no need for “autosave”, because all changes are stored immediately to
disk, so a program crash doesn't lose \emph{any} information.
Instead of saving at the end of the session the current branch is merged into
master, and this isn't even related to shutting down the computer (i.\,e.\ the session can span an arbitrary amount of time).
\item I'm sure one can make good use of branching to design workflows with
“named sessions” (e.\,g.\ for trying out some hacks or to switch working context
temporarily).
\item And of course this opens the door wide for \emph{collaborative
workflows} where, for example, the musicians can evaluate takes while the editor
is preparing his editing plan.
\end{itemize}

As we are talking of “atomic” commits the risk of running into merge conflicts
at any point is rather small.
In a local set-up they could only happen when two or more “sessions” (i.\,e.\
branches) are open at the same time.
When working with remote repositories and multiple participants chances of
running into conflicts are higher, but as all changes are written
programmatically we know pretty well what is going on and can design simple and
clean ways to resolve any conflicts.

However, this has to be absolutely transparent, and the ordinary user should not
be required to learn \emph{any} versioning syntax or terminology as
such a requirement would significantly reduce acceptance with the main target
group.

\medskip
Instead of inventing a new plain text file format or using some form of
\textsc{xml} I suggest using LilyPond's%
\footnote{\url{http://www.lilypond.org}}
file format for storing take files.
LilyPond is a plain text based score engraving application and offers a
concise input language for describing musical structures.
This language is very usable for version control because it has
significantly less overhead than, say, \textsc{xml} formats.
It is very straightforward to (programmatically) write a LilyPond file with the
skeleton timing structure of the composition, while one can insert any
kind of annotations through commands at \emph{musical} moments.
These commands can then, for example, link the musical moment to an
exact file position.
Another possible application would be to use LilyPond to “engrave” the
editing plan as a rhythmic score, indicating the used takes, their
transitions or any additional comments.

\section{Outlining an Application}

As stated earlier I don't have any concrete plans with this, but I will outline
a concept for a \textsc{gui} application as if I were going to start development
soon.
Of course such a project can be realized with a wide range of tools, but if I
were to start it I would go for a PyQt application.
Having some experience contributing to the Frescobaldi%
\footnote{\url{http://www.frescobaldi.org}}
LilyPond editor, I think it is a very comfortable platform to work with -- and
particularly the one I'm familiar with. Maybe it is not the perfect solution
for developing \textsc{dsp} software, but as my approach is mostly about
\emph{organizing} stuff and the user interface this shouldn't be much of an issue.
And hopefully there are ways to interact with code performing better for issues
such as high-precision audio playback.

The main design issue I don't have a clear opinion about is the \emph{scope} of
an application.
The only possible target in my eyes would be a tool that is used by
professional recording producers -- creating yet another toy \textsc{daw} seems
waste of time.
But while it seems fascinating to have a comprehensive \textsc{daw} using my new
concepts it would probably be grossly negligent to try competing with all the
existing \textsc{dsp} knowledge that already is on the market.
I currently see three options to deal with this situation:

\begin{itemize*}
\item Find existing code that can be exploited, either in dedicated libraries or
in open source software.
Of course this could limit the choice of programming languages.
\item Find an existing project that would be interested in incorporating the
ideas into their \textsc{daw} system.
This would probably limit the choice of languages even more, but might
compensate for this with existing experience and a framework.
\item Design the program as a standalone complementary tool, just like you'd
have a bibliographic or indexing tool for authoring scholarly books.
This seems the most versatile option, and as it's the only one that is currently
conceivable from my position I'll continue this paper with this option.
\end{itemize*}


\subsection{Intended Scope}

An application should provide a comprehensive set of tools to cover the whole
process of managing the recorded material and preparing an editing plan.
There are other parts of the production process that seem too ambitious to
tackle individually, and these should rather be realized by integrating existing
tools as seamlessly as possible.
Amongst those are most notably the whole field of digital signal processing,
particularly recording/mixing and crossfade editing.
However, I will give these aspects some thought at the appropriate places.

The tasks our projected application should be able to perform and that I'll
describe in the following sections are:
Accessing/importing and managing audio files,
annotating individual takes and the project structure, and
preparing and exporting the editing plan.

\subsection{Managing Audio Files}

The starting point for working on a musical production is the recorded audio
material.
Our projected application would \emph{not} (at least for a start) try to
implement a complete recording solution, instead we are going to simply
\emph{use} existing audio material.

The simplest way to do so is to manage audio files by incorporating them through
a symbolik link.
Combined with file system monitoring this would have the main advantage that the
recording producer can use whatever tools she is used to (or has available).
When a file has been recorded within the other application it is automatically
available in the pool of our application.

Of course it would be nice to be able to control recording from within our
application, and probably it would also be manageable to implement a way to do
this, e.\,g.\ using \texttt{arecord} and passing on the responsibility for audio
setup to something else.
Another option -- although more restricted than desirable -- would be to use
\textsc{osc} to remotely control \emph{Ardour}.

There is one thing to note here: Our program conceptually deals with audio
mainly independently of the number of parallel tracks -- a take is actually treated like
a monophonic region.
I'm not sure if the program should really make a stereo mixdown for its working
files or if it actually should playback a mix of all existing tracks.

\section{Components of the Application}

\subsection{Annotating a Single Take}
The core element of our application is an editor for individual takes.
This is what we'll spend most of our time with, particularly during the initial
evaluation of the recorded material.
Therefore this shouldn't be a dialog but a dockable panel which is permanently
visible and simply gets updated with the currently focused take.
It contains a zoomable graphical representation of the audio in the take, very
much like regions are usually displayed in \textsc{daw}s.
Playback controls should cover the usual range of features like moving forward
and backward, listening to ranges, starting from marker positions etc.

This representation of a take is covered by a timeline which reflects the
\emph{musical} structure.
That is it doesn't (primarily) display a time code but rather barlines, rests
and time signatures, similar to a percussion staff.
Rehearsal marks, tempo indications and more may easily be added to this.
Initially (when we don't know more than the mere number of measures in the take)
this musical structure consists of evenly spaced measures with a default time
signature, but during work the editor can add relevant information such as time
signature changes.
And she can set markers linking the musical structure to positions in the audio
files.
Through this the program also gets an inherent understanding of tempo changes,
allowing it to interpolate positions for playback.
The timeline should visually distinguish between \emph{explicitly set} and
\emph{interpolated} positions.

Information edited in an individual take will be propagated to the whole project
and may be averaged with information from other takes.
This will deepen the project's knowledge about the musical structure and
gradually improve the accuracy of position interpolations.
(You will recall that one of our goals is to make it possible to \emph{directly}
access specific musical positions in takes.)
This means that when editing any given take, the program will also take all the
timing information into account that has been set in other takes covering the
same musical range.

But of course the most relevant information added to takes is our evaluation.
It should be possible to assign ratings to any range or spot.
These ratings are given numerically, with some special values available for
marking something as “unusable”, “usable as base material”, “suggested” or ”to
be used” and “editing plan”.
Apart from numerical or tabular display the waveform representation could
visualize this through differently colored background or similar means.
Adding ratings should be accessible through right-clicking on the waveform or
through keyboard shortcuts while playing back.
Apart from the ratings, one can add arbitrary verbal comments to the take as a
whole, specific spots or regions or explicit ratings.

These ratings are stored in the LilyPond files with new appropriately designed
commands.
The \emph{musical} position is defined by the position in the input file, while
the position in the audio file is a mandatory argument of the commands.
This will ensure that the spots that are of particular interest are linked
especially thoroughly.

It should be possible (but I'm not sure yet if it's worth the effort) to compile
a score fragment with LilyPond, representing the musical structure and
displaying the rating and comments in some way.


\subsection{Take List Overview}
Our application should have a panel that is quite similar to a \term{region
list}, but with some unique features.
While traditional region lists display the project's regions sorted by the
takes' original order or maybe by their position in the arrangement, we have
superior information available for this purpose, namely the information about the
\emph{musical} content and our ratings.

The default sort order for takes in this list view would be their (musical)
starting point, but depending on the current working perspective we can use
different parameters for \emph{sorting} and \emph{filtering}.
For example, when we're interested in a given musical moment we can hide all
takes that don't cover this moment, and sort the remaining ones by their rating
in descending order.
We can display the interesting moment as a cursor line and visualize the ratings
-- this way we have an instant visual idea about the situation and possible
alternative takes.

I won't list more ideas here (because that's something for the actual
application design), but I'm sure you get the idea how this approach can
dramatically increase productivity for the main task a recording producer has
for editing classical music: preparing the editing plan.

\subsection{[Score Viewer]}
Optionally our application could contain a \term{score viewer}.
While this isn't essential for the application to work at all, it is very much in
line with its concept and would increase the benefits of the new approach.
As one key aspect of the new approach is the awareness of the musical context,
it's very natural to make use of that for score viewing too.

The editor can import the (annotated) score into the project as a \textsc{pdf} or as a
series of image files.
In a (very short) editing round she would have to annotate that score, providing
reference points for system changes combined with bar numbers.
That way the score viewer can easily display the score fragment corresponding to
the currently edited take or region.

\subsection{Editing Plan}
Conceptually an editing plan is a list of pairs linking musical sections to
takes to be used.
Traditionally this is reflected by placing cropped takes at the appropriate
position in the arranger window.
There are two issues with this that can be addressed much better in our new
application:
Of course it is much more natural to access the takes based on their musical
content than simply referencing their index number and file positions.
And as mentioned earlier we won't \emph{place} takes on a fixed timeline but
rather \emph{chain} them, which makes a huge difference.
We will also call cropped selections from takes \term{regions} but will treat them
differently from traditional \textsc{daw}s.

\medskip
The second main component of our application is the \term{editing plan} window
-- which is the replacement for the former \term{arranger window}.
It is similar in that it displays a timeline and multiple tracks, but both are
conceived significantly differently.

The timeline is organized by the musical structure of the music instead of a
steady timecode.
There \emph{is} a timecode display, but this is merely the volatile result of
the current state of the arrangement.
The playing position of any given region isn't absolute, but relative to the
previous region.
This is what I call \emph{chaining} the regions, and this will ensure that
modifications won't have any negative impact on the regions following later in
the timeline.

The topmost track is the \term{editing plan}, and having this completely filled
is our ultimate goal.
Initially it's empty, but the display will already be scaled using the timing
information stored in the take files.

Below this is a group of tracks reserved for regions with a certain minimum
rating -- this is the \term{dashboard} in which to play around and choose the finally
used regions.
The bottom part of the window will be populated by the remaining takes or
regions.

Note that all tracks except the \term{editing plan} are populated automatically, based
on the rating we have given the individual takes.
We will have a cursor indicating the editing position in the piece, and
whenever this changes the tracks will be updated with the takes that contain
this position.
Takes are presented as blocks and display the rating through coloring.
This way one always has a complete overview of the available recorded material,
structured by relevance and quality through the assignment to tracks and by the
color-coding.
And in particular it is immediately obvious when there are musical sections we
haven't found suitable takes for (because the respective parts of the
dashboard are empty).

Of course it is always possible to listen to any single take from this window,
and whenever a take/region is selected the take is loaded into the \term{take
annotation} tool.
But the tool that is used to trim and fine-tune the regions to be used in the
editing plan is the \term{transition editor}

\subsubsection{Transition Editor}

One crucial point when preparing an editing plan is not only to select usable
takes for a given musical moment or range but also to check if any considered
transition between takes is actually doable.
Takes have to be compatible with respect to tempo, dynamics, internal balance
and “atmosphere” (reverberation or piano pedal trails are particularly tricky)
-- and this can only be tested by actually listening to a transition.
One could even conceive an editing plan as defined by a series of transitions
instead of a series of regions.

And this is the way our application will “think” like.
The editing plan isn't improved by placing regions on the topmost track but
rather by applying transitions which have been defined in the \term{transition
editor}.

This will be a tool (a modal dialog or a dockable panel) that displays two
regions where the recording producer can edit the transition between them.
Two regions are placed around a centered cursor representing a musical moment.
Usually we will consider one of the regions as “given” and look for a matching
region, but essentially the core of the transition is the musical moment.

The tool will now offer to choose from a list of available takes sorted or filtered
by rating, “available” meaning that they contain the musical moment.
It should also be possible to see how a take develops, that is, how the rating
changes over time.
This way the editor will be able to determine if a take can be used as a short
insert only or also as a base take for a longer section.

The tool will of course make use of the timing information stored in the takes,
so the accuracy of the suggested material will improve over editing time.
This means that when selecting a take to try a transition it will show
approximately the right position out of the box!
Usually we will now search for the right position, and then we can save this as
additional information, increasing the coverage of the whole project.
I expect that this will lead to near-instant matches in a very short time.

\paragraph{Note:}
The \term{transition editor} will provide a crossfade editor to fine-tune the
transition.
It will be crucial to see to what extent this tool can be made to match
professional needs.
If this \emph{is} possible, our application could aim at being a solution for the
whole process up to the final cut, either exporting directly to CD or routing to
an external mixing/mastering application.
In that case I'd suggest an approach that normally treats the musical material
as \emph{one} conceptual track and offering multitrack access to the transitions
of individual audio tracks only on special demand.
If it is \emph{not} possible to head for a professional editing tool, we would be
restricted to \emph{preparing} the editing plan and exporting it to another
editing application.
In either case I assume from my experience that a simple crossfade will in most
cases at least be sufficient to judge if a given transition can be applied at
all.

\medskip
The key feature of this tool is that it is completely independent of the
project's timeline but simply considers a relation of two takes.
The result of such an edit will be a \term{transition} object which can be used
in the editing plan.
Other than in the traditional arranger window these transition objects can be
stored for later reuse.
This means that when trying a different transition you don't have to discard any
previous work -- which you'd have to in the arranger window.
Instead the project knows which transitions can be applied to any given take.
Probably it will be possible to simply drag\,\&\,drop a transition on the
editing plan to use it there, making the whole process of juggling with possible
arrangements \emph{significantly} easier than it is nowadays.

\subsection{Exporting Your Work}

Once we have edited all necessary transitions to join the take fragments to a
coherent editing plan, everything is there to export: the editing plan itself as well
as a rendered “preview” audio file or even a “production” audio file.
The possible options will heavily depend on programming capabilities and the scope
a project may get. This subsection provides some ideas on the different
possibilities.

One thing that can always be exported is the editing plan itself. This will contain
a list of transition points (in musical terms), used audio files and sample-accurate
positions in the files.
This list can be exported in several formats, the most traditional way being a written
list, e.\,g.\ in \textsc{pdf}, word processor or plain text/Markdown format.
Optionally LilyPond can be used to produce the editing plan as a score,
presumably in form of a percussion-like staff with additional information
processed in a visually appealing way.
This will enable the recording producer to import the files into her \textsc{daw} of
choice and arrange them for editing with the least amount of lookup overhead.

While this would make the choice of tools most independent it surely isn't very
elegant to require the producer to manually add and arrange regions in a second
application. Therefore a step to provide a more integrated workflow would be to
directly export to project files of external \textsc{daw} applications, for example
\emph{Ardour}. This would add all used take files to the audio pool of the other
application's project and lay the regions out in a way that they can be edited
with minimal overhead. This sounds like a very good solution because it offers
great flexibility by allowing the producer to use all the mixing, mastering and
automation tools of the other application. The only issue is to propagate any
post-export changes to the editing plan to the other application. While this
\emph{may} be possible it can also be very complex, depending on the other
application's file format.

The same restriction is true for exporting rendered audio files with all transitions
applied. These could either be saved to disk (as discrete per-track audio files) or
routed to the output (e.\,g. Jack) to be used in another application. If the new
application's transition editor can be made professional enough this would be
the preferred solution for the time being.

It has to be stressed that this limitation isn't actually a limitation of the new
concept but a limitation that emerges when interfacing with the old concept by
exporting to applications that use the fixed timeline model. Therefore the
ultimate solution would be to add all the functionality to the new application
that a recording producer might need to professionally prepare a recording of
classical music. Obviously this is a very ambitious idea, and not a really realistic
one -- because it would basically mean to create a powerful application from scratch
that can rival tools like \emph{Ardour} or \emph{ProTools} in their native domain.
Therefore the goal should really be to find original solutions to streamline the
interface to the lowlevel \textsc{dsp} \emph{processing} tools while keeping the new program on a
higher, conceptual level of a user interface to \emph{manage} recorded material.



\section{Collaborative Workflows}

Earlier I mentioned that I intend to use LilyPond, i.\,e.\ plain text files, as
the storage format and Git to manage the editing process.
While this offers big advantages by itself, it can even enable completely new
collaborative workflows.
As they are new, people aren't used to them so far, and I can foresee recording
producers expressing reservations against them.
But I'm sure they can result in improvements as they do in editing scores -- where
reservations are similarly high.

As the file storage is inside a Git repository it's perfectly possible to make
use of Git's distributed nature to share a project through a central repository
on a server (or with a service provider).
Probably one would keep the audio material out of that repository and distribute
it separately, but all the editing can be shared with all of Git's convenience.
For example, musicians can evaluate the raw material and add their ratings to
takes, adding their opinion and expertise to the process.
At the same time the recording producer can make preliminary states of editing
available to the musicians to get their feedback early in the process.
One would have to create a certain level of role-based privileges, but that's an
issue for a concrete application design phase.
It would be nice if musicians could create (or sketch) possible transitions that
the recording producer then could use (or decide not to use).

I know such a workflow requires a fair amount of mutual trust and would force the
recording producer to give away a certain level of control.
But in my experience such collaborative workflows are \emph{very} fruitful and
enriching%
\footnote{See \url{http://lilypondblog.org/?p=1380}}.


\section{Conclusions}

TBD (after initial review)


\section{Acknowledgments}

My thanks go to \ldots .

\end{document}

